\documentclass[nocrop]{sesamanuel}
% ============================================================================================
% ======= Générer les correction dans un dossier spécifique
% ============================================================================================

\renewcommand\PrefixeCorrection{corrections/}

% ============================================================================================
\usepackage{sesamanuelTIKZ}

\begin{document}
% ============================================================================================
% ======= Thèmes personnalisés
%
% \NewThema{N}{n}{titre}{Titre}{TITRE}{couleur entete et ...}{couleur pied de page et ...}
%
% ============================================================================================
\NewThema{N}{n}{nombres \&\\calculs}{Nombres \&\\calculsManuel}{NOMBRES \&\\CALCULS}{B1}{B1!50}

% Pour la géométrie on garde le théme d'origine

\NewThema{D}{d}{organisation et \\et gestion de données}{Organisation et \\et gestion de données}{ORGANISATION ET\\GESTION DE DONNÉES}{PartieStatistique}{PartieStatistique!50}

\NewThema{M}{m}{grandeurs\\et mesures}{Grandeurs\\et mesures}{GRANDEURS\\ET MESURES}{G1}{G1!50}

\NewThema{A}{a}{algorithmique et programmation}{Algorithmique et programmation}{ALGORITHMIQUE ET PROGRAMMATION}{J1}{J1!50}
% ============================================================================================
% ======= Tests
% ============================================================================================

\themaN
\chapter{themaN1}
\chapter{themaN2}

\themaD
\chapter{themaD}
\chapter{themaD1}

\themaG
\chapter{themaG}
\chapter{themaG1}
\chapter{themaG2}

\themaM
\chapter{themaM}
\chapter{themaM1}

\themaA
\chapter{themaA}
\chapter{themaA1}
\chapter{themaA2}

% ============================================================================================
% ======= Thèmes de base
% ============================================================================================

%\themaG
%\chapter{themaG}
%
%\themaF
%\chapter{themaF}
%
%\themaS
%\chapter{themaS}

\end{document}
