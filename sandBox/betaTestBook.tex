% ============================================================================================
% ======= On force le passage des options au paquet xcolor pour l'utilisation
% ======= du paquet profcollege
% ============================================================================================
\PassOptionsToPackage{table}{xcolor}
\PassOptionsToPackage{svgnames}{xcolor}
\documentclass[nocrop]{sesamanuel}
% ============================================================================================
% ======= Générer les correction dans un dossier spécifique
% ============================================================================================

\renewcommand\PrefixeCorrection{corrections/}

% ============================================================================================
\usepackage{sesamanuelTIKZ}
\usepackage{ProfCollegeNewCAN}

\begin{document}
% ============================================================================================
% ======= Thèmes personnalisés
%
% \NewThema{N}{n}{titre}{Titre}{TITRE}{couleur entete et ...}{couleur pied de page et ...}
%
% ============================================================================================
\NewThema{N}{n}{nombres \&\\calculs}{Nombres \&\\calculsManuel}{NOMBRES \&\\CALCULS}{B1}{B1!50}

% Pour la géométrie on garde le théme d'origine

\NewThema{D}{d}{organisation et \\et gestion de données}{Organisation et \\et gestion de données}{ORGANISATION ET\\GESTION DE DONNÉES}{PartieStatistique}{PartieStatistique!50}

\NewThema{M}{m}{grandeurs\\et mesures}{Grandeurs\\et mesures}{GRANDEURS\\ET MESURES}{G1}{G1!50}

\NewThema{A}{a}{algorithmique et programmation}{Algorithmique et programmation}{ALGORITHMIQUE ET PROGRAMMATION}{J1}{J1!50}
% ============================================================================================
% ======= Tests
% ============================================================================================

\themaN
\chapter{themaN1}
% ============================================================================================
% ======= Rappels de connaissances et série de petits exercices
% ============================================================================================

\begin{prerequis}
    Liste de prérequis - Ici le tritre est le titre par défaut
    \begin{itemize}
    \item prérequis 1
    \item prérequis 2
    \item prérequis 3
    \item prérequis 4    
    \end{itemize}
\end{prerequis}

\begin{prerequis}[Titre prérequis modifié]
    Liste de prérequis - Ici le tritre est modifié
    \begin{itemize}
    \item prérequis 1
    \item prérequis 2
    \item prérequis 3
    \item prérequis 4    
    \end{itemize}
\end{prerequis}


\begin{autoeval}
    \begin{multicols}{2}
      \begin{exercice}
        Ex1
      \end{exercice}
      \begin{corrige}
        corEx1
      \end{corrige}
      \begin{exercice}
        Ex2
      \end{exercice}
      \begin{corrige}
        corEx2
      \end{corrige}
  \vfill \columnbreak
      \begin{exercice}
        Ex3
      \end{exercice}
      \begin{corrige}
        corEx3
      \end{corrige}
    \end{multicols}
  \end{autoeval}
%\chapter{themaN2}

\themaD
\chapter{themaD1}
%\chapter{themaD2}

\themaG
\chapter{themaG1}
%\chapter{themaG2}
%\chapter{themaG3}

\themaM
\chapter{themaM1}
%\chapter{themaM2}

\themaA
\chapter{themaA1}
%\chapter{themaA2}
%\chapter{themaA3}

% ============================================================================================
% ======= Thèmes de base
% ============================================================================================

%\themaG
%\chapter{themaG}
%
%\themaF
%\chapter{themaF}
%
%\themaS
%\chapter{themaS}

\end{document}
